\documentclass[landscape,twocolumn,letterpaper]{article}

\usepackage[document]{ragged2e}
\usepackage[top=2cm, bottom=2cm, left=1.5cm, right=1.5cm]{geometry}
\usepackage{graphicx}
\setlength{\columnsep}{40pt}

\renewcommand*\familydefault{\ttdefault} %% Only if the base font of the document is to be typewriter style

\begin{document}

\thispagestyle{empty}
\pagestyle{empty}

\tiny.
\normalsize
\vfill
\centering{typeset in \LaTeX by dymkave}
\eject

\tiny.
\normalsize
\vspace{30 mm}

\begin{center}
  \includegraphics[scale=0.4]{bus.jpg}
  \section*{\centering{Fuck Off Google}}
  \subsection*{\centering{Invisible Committee}}    
\end{center}


\newpage
\clearpage

\tiny.
\normalsize
\vfill\eject

\tiny.
\normalsize
\vspace{40 mm}

\section*{\centering{Fuck Off Google}}
\vspace{5 mm}

1. There are no “Facebook revolutions”, but
there is a new science of government,
cybernetics. 2. War against all things smart!
The Poverty of Cybernetics. 4. Techniques
against Technology.

\vspace{10 mm}

\subsection*{1. There are no “Facebook
revolutions”, but there is a new science of
government, cybernetics.}
\vspace{5 mm}

\raggedright

The genealogy is not well known, and it deserves to be. Twitter
descends from a program named TXTMob, invented by American activists
as a way to coordinate via cellphones during protests against the
Republican National Convention in 2004. The application was used by
some 5000 people to share real-time information about the different
actions and movements of the police. Twitter, launched two years
later, was used for similar purposes, in Moldova for example, and the
Iranian demonstrations of 2009 popularized the idea that it was the
tool for coordinating insurgents, particularly against the
dictatorships. In 2011, when rioting reached an England thought to be
definitively impassive, some journalists were sure that tweeting had
helped spread the disturbances from their epicenter,
Tottenham. Logical, but it turned out that for their communication
needs the rioters had gone with BlackBerry, whose secure telephones
had been designed for the upper management of banks and
multinationals, and the British secret service didn’t even have the
decryption keys for them. Moreover, a group of hackers hacked into
BlackBerry’s site to dissuade the company from cooperating with the
police in the aftermath. If Twitter enabled a self-organization on
this occasion it was more that of the citizen sweepers who volunteered
to sweep up and repair the damage caused by the confrontations and
looting. That effort was relayed and coordinated by CrisisCommons, a
“global network of volunteers working together to build and use
technology tools to help respond to disasters and improve resiliency
and response before a crisis.” At the time, a French left-wing rag
compared this undertaking to the organization of the Puerta del Sol
during the Indignants Movement, as it’s called. The comparison between
an initiative aimed at a quick return to order and the fact of several
thousand people organizing to live on an occupied plaza, in the face
of repeated assaults by the police, may look absurd. Unless we see in
them just two spontaneous, connected, civic gestures. From 15-M on,
the Spanish “indignados,” a good number of them at least, called
attention to their faith in a citizens’ utopia. For them the digital
social networks had not only accelerated the spread of the 2011
movement, but also and more importantly had set the terms of a new
type of political organization, for the struggle and for society: a
connected, participatory, transparent democracy. It’s bound to be
upsetting for “revolutionaries” to share such an idea with Jared
Cohen, the American government’s anti-terrorism adviser who contacted
Twitter during the “Iranian revolution” of 2009 and urged them to
maintain it’s functioning despite censorship. Jared Cohen has recently
cowritten with Google’s former CEO, Eric Schmidt, a creepy political
book, The New Digital Age. On its first page one reads this misleading
sentence: “The Internet is the largest experiment involving anarchy in
history.” “In Tripoli, Tottenham or Wall Street people have been
protesting failed policies and the meager possibilities afforded by
the electoral system… They have lost faith in government and other
centralized institutions of power… There is no viable justification
for a democratic system in which public participation is limited to
voting. We live in a world in which ordinary people write Wikipedia;
spend their evenings moving a telescope via the Internet and making
discoveries half a world away; get online to help organize a protest
in cyberspace and in the physical world, such as the revolutions in
Egypt or Tunisia or the demonstrations of the the ‘indignados’
throughout Spain; or pore over the cables revealed by WikiLeaks. The
same technologies enabling us to work together at a distance are
creating the expectation to do better at governing ourselves.” This is
not an “indignada”speaking, or if so, she’s one who camped for a long
time in an office of the White House: Beth Noveck directed the “Open
Government Initiative” of the Obama administration. That program
starts from the premise that the governmental function should consist
in linking up citizens and making available information that’s now
held inside the bureaucratic machine. Thus, according to New York’s
city hall, “the hierarchical structure based on the notion that the
government knows what’s good for you is outdated. The new model for
this century depends on co-creation and collaboration.”

Unsurprisingly, the concept of Open Government Data was formulated not
by politicians but by computer programmers – fervent defenders of open
source software development, moreover – who invoked the U.S. founding
fathers’ conviction that “every citizen should take part in
government.” Here the government is reduced to the role of team leader
or facilitator, ultimately to that of a “platform for coordinating
citizen action.”  The parallel with social networks is fully
embraced. “How can the city think of itself in the same way Facebook
has an API ecosystem or Twitter does?” is the question on their minds
at the New York mayor’s office. “This can enable us to produce a more
user-centric experience of government. It’s not just the consumption
but the co-production of government services and democracy.” Even if
these declarations are seen as fanciful cogitations, as products of
the somewhat overheated brains of Silicon Valley, they still confirm
that the practice of government is less and less identified with state
sovereignty. In the era of networks, governing means ensuring the
interconnection of people, objects, and machines as well as the free –
i.e., transparent and controllable—circulation of information that is
generated in this manner. This is an activity already conducted
largely outside the state apparatuses, even if the latter try by every
means to maintain control of it. It’s becoming clear that Facebook is
not so much the model of a new form of government as its reality
already in operation. The fact that revolutionaries employed it and
still employ it to link up in the street en masse only proves that
it’s possible, in some places, to use Facebook against itself, against
its essential function, which is policing. When computer scientists
gain entry, as they’re doing, into the presidential palaces and
mayors’ offices of the world’s largest cities, it’s not so much to set
up shop as it is to explain the new rules of the game: government
administrations are now competing with alternative providers of the
same services who, unfortunately for them, are several steps
ahead. Suggesting their cloud as a way to shelter government services
from revolutions –services like the land registry, soon to be
available as a smartphone application– the authors of The New Digital
Age inform us and them: “In the future, people won’t just back up
their data; they’ll back up their government.” And in case it’s not
quite clear who the boss is now, it concludes: “Governments may
collapse and wars can destroy physical infrastructure but virtual
institutions will survive.” With Google, what is concealed beneath the
exterior of an innocent interface and a very effective search engine,
is an explicitly political project. An enterprise that maps the planet
Earth, sending its teams into every street of every one of its towns,
cannot have purely commercial aims. One never maps a territory that
one doesn’t contemplate appropriating. “Don’t be evil!”: let yourself
go. It’s a little troubling to note that under the tents that covered
Zucotti Park and in the offices of planning –a little higher in the
New York sky—the response to disaster is conceived in the same terms:
connection, networking, self-organization. This is a sign that at the
same time that the new communication technologies were put into place
that would not only weave their web over the Earth but form the very
texture of the world in which we live, a certain way of thinking and
of governing was in the process of winning. Now, the basic principles
of this new science of government were framed by the same ones,
engineers and scientists, who invented the technical means of its
application. The history is as follows. In the 1940’s, while he was
finishing his work for the American army, the mathematician Norbert
Wiener undertook to establish both a new science and a new definition
of man, of his relationship with the world and with himself. Claude
Shannon, an engineer at Bell and M.I.T., whose work on sampling theory
contributed to the development of telecommunications, took part in
this project. As did the amazing Gregory Bateson, a Harvard
anthropologist, employed by the American secret service in Southeast
Asia during the Second World War, a sophisticated fan of LSD and
founder of the Palo Alto School. And there was the truculent John von
Neumann, writer of the First Draft of a Report on the EDVAC, regarded
as the founding text of computer science – the inventor of game
theory, a decisive contribution to neoliberal economics – a proponent
of a preventive nuclear strike against the U.S.S.R., and who, after
having determined the optimal points for releasing the Bomb on Japan,
never tired of rendering various services to the American army and the
budding C.I.A. Hence the very persons who made substantial
contributions to the new means of communication and to data processing
after the Second World War also laid the basis of that “science” that
Wiener called “cybernetics.” A term that Ampère, a century before, had
had the good idea of defining as the “science of government.” So we’re
talking about an art of governing whose formative moments are almost
forgotten but whose concepts branched their way underground, feeding
into information technology as much as biology, artificial
intelligence, management, or the cognitive sciences, at the same time
as the cables were strung one after the other over the whole surface
of the globe. We’re not undergoing, since 2008, an abrupt and
unexpected “economic crisis,” we’re only witnessing the slow collapse
of political economy as an art of governing. Economics has never been
a reality or a science; from its inception in the 17th century, it’s
never been anything but an art of governing populations. Scarcity had
to be avoided if riots were to be avoided – hence the importance of
“grains” – and wealth was to be produced to increase the power of the
sovereign. “The surest way for all government is to rely on the
interests of men,” said Hamilton. Once the “natural” laws of economy
were elucidated, governing meant letting its harmonious mechanism
operate freely and moving men by manipulating their
interests. Harmony, the predictability of behaviors, a radiant future,
an assumed rationality of the actors: all this implied a certain
trust, the ability to “give credit.”  Now, it’s precisely these tenets
of the old governmental practice which management through permanent
crisis is pulverizing. We’re not experiencing a “crisis of trust” but
the end of trust, which has become superfluous to government. Where
control and transparency reign, where the subjects’ behavior is
anticipated in real time through the algorithmic processing of a mass
of available data about them, there’s no more need to trust them or
for them to trust. It’s sufficient that they be sufficiently
monitored. As Lenin said, “Trust is good, control is better.”

The West’s crisis of trust in itself, in its knowledge, in its
language, in its reason, in its liberalism, in its subject and the
world, actually dates back to the end of the 19th century; it breaks
forth in every domain with and around the First World War. Cybernetics
developed on that open wound of modernity. It asserted itself as a
remedy for the existential and thus governmental crisis of the
West. As Norbert Wiener saw it, “We are shipwrecked passengers on a
doomed planet. Yet even in a shipwreck, human decencies and human
values do not necessarily vanish, and we must make the most of
them. We shall go down, but let it be in a manner to which we may look
forward as worthy of our dignity”. Cybernetic government is inherently
apocalyptic. Its purpose is to locally impede the spontaneously
entropic, chaotic movement of the world and to ensure “enclaves of
order,” of stability, and – who knows? – the perpetual self-regulation
of systems, through the unrestrained, transparent, and controllable
circulation of information. “Communication is the cement of society
and those whose work consists in keeping the channels of communication
open are the ones on whom the continuance or downfall of our
civilization largely depends,” declared Wiener, believing he knew. As
in every period of transition, the changeover from the old economic
governmentality to cybernetics includes a phase of instability, a
historical opening where governmentality as such can be put in check.

\subsection*{2. War against all things smart!}


In the 1980’s, Terry Winograd, the mentor of Larry Page, one of the
founders of Google, and Fernando Flores, the former finance minister
of Salvador Allende, wrote concerning design in information technology
that “the most important designing is ontological. It constitutes an
intervention in the background of our heritage, growing out of our
already existent ways of being in the world, and deeply affecting the
kinds of beings that we are…It is necessarily reflective and
political.” The same can be said of cybernetics. Officially, we
continue to be governed by the old dualistic Western paradigm where
there is the subject and the world, the individual and society, men
and machines, the mind and the body, the living and the
nonliving. These are distinctions that are still generally taken to be
valid. In reality, cybernetized capitalism does practice an ontology,
and hence an anthropology, whose key elements are reserved for its
initiates. The rational Western subject, aspiring to master the world
and governable thereby, gives way to the cybernetic conception of a
being without an interiority, of a selfless self, an emergent,
climatic being, constituted by its exteriority, by its relations. A
being which, armed with its Apple Watch, comes to understand itself
entirely on the basis of external data, the statistics that each of
its behaviors generates. A Quantified Self that is willing to monitor,
measure, and desperately optimize every one of its gestures and each
of its affects. For the most advanced cybernetics, there’s already no
longer man and his environment, but a system-being which is itself
part of an ensemble of complex information systems, hubs of autonomic
processes – a being that can be better explained by starting from the
middle way of Indian Buddhism than from Descartes. “For man, being
alive means the same thing as participating in a broad global system
of communication”, asserted Wiener in 1948. Just as political economy
produced a homo economicus manageable in the framework of industrial
States, cybernetics is producing its own humanity. A transparent
humanity, emptied out by the very flows that traverse it, electrified
by information, attached to the world by an ever-growing quantity of
apparatuses. A humanity that’s inseparable from its technological
environment because it is constituted, and thus driven, by that. Such
is the object of government now: no longer man or his interests, but
his “social environment”. An environment whose model is the smart
city. Smart because by means of its sensors it produces information
whose processing in real time makes self-management possible. And
smart because it produces and is produced by smart
inhabitants. Political economy reigned over beings by leaving them
free to pursue their interest; cybernetics controls them by leaving
them free to communicate. “We need to reinvent the social systems in a
controlled framework,” according to M.I.T. professor Alex Pentland, in
an article from 2011. The most petrifying and most realistic vision of
the metropolis to come is not found in the brochures that IBM
distributes to municipalities to sell them software for managing the
flows of water, electricity, or road traffic. It’s rather the one
developed in principle “against” that Orwellian vision of the city:
“smarter cities” coproduced by their residents themselves (in any case
by the best connected among them). Another M.I.T. professor traveling
in Catalonia is pleased to see its capital becoming little by little a
“fab city”: “Sitting here right in the heart of Barcelona I see a new
city being invented where everyone will have access to the tools to
make it completely autonomous” The citizens are thus no longer
subalterns but smart people, “receivers and generators of ideas,
services, and solutions,” as one of them says. In this vision, the
metropolis doesn’t become smart through the decision-making and action
of a central government, but appears, as a “spontaneous order”, when
its inhabitants “find new ways of producing, connecting, and giving
meaning to their own data.” The resilient metropolis thus emerges, one
that can resist every disaster. Behind the futuristic promise of a
world of fully linked people and objects, when cars, fridges, watches,
vacuums, and dildos are directly connected to each other and to the
Internet, there is what is already here: the fact that the most
polyvalent of sensors is already in operation: myself. “I” share my
geolocation, my mood, my opinions, my account of what I saw today that
was awesome or awesomely banal. I ran, so I immediately shared my
route, my time, my performance numbers and their self-evaluation. I
always post photos of my vacations, my evenings, my riots, my
colleagues, of what I’m going to eat and who I’m going to fuck. I
appear not to do much and yet I produce a steady stream of
data. Whether I work or not, my everyday life, as a stock of
information, remains fully valuable. “Thanks to the widespread
networks of sensors, we will have a God’s eye view of ourselves. For
the first time, we can precisely map the behavior of masses of people
at the level of their daily lives,” enthuses one of the
professors. The great refrigerated storehouses of data are the pantry
of current government. In its rummaging through the databases produced
and continuously updated by the everyday life of connected humans, it
looks for the correlations it can use to establish not universal laws
nor even “whys,” but rather “whens” and “whats,” onetime, situated
predictions, not to say oracles. The stated ambition of cybernetics is
to manage the unforeseeable, and to govern the ungovernable instead of
trying to destroy it. The question of cybernetic government is not
only, as in the era of political economy, to anticipate in order to
plan the action to take, but also to act directly upon the virtual, to
structure the possibilities. A few years ago, the LAPD bought itself a
new software program called PredPol. Based on a heap of crime
statistics, it calculates the probabilities that a particular crime
will be committed, neighborhood by neighborhood, street by
street. Given these probabilities updated in real time, the program
itself organizes the police patrols in the city. A founder
cybernetician wrote in Le Monde in 1948: “We can dream of a time when
the machine à gouverner will – for good or evil, who knows?  –
compensate for the shortcomings, obvious today, of the leaders and
customary apparatuses of politics.” Every epoch dreams the next one,
even if the dream of the one may become the daily nightmare of the
other. The object of the great harvest of personal information is not
an individualized tracking of the whole population. If the
surveillants insinuate themselves into the intimate lives of each and
every person, it’s not so much to construct individual files as to
assemble massive databases that make numerical sense. It is more
efficient to correlate the shared characteristics of individuals in a
multitude of “profiles,” with the probable developments they
suggest. One is not interested in the individual, present and entire,
but only in what makes it possible to determine their potential lines
of flight. The advantage of applying the surveillance to profiles,
“events,” and virtualities is that statistical entities don’t take
offense, and individuals can still claim they’re not being monitored,
at least not personally. While cybernetic governmentality already
operates in terms of a completely new logic, its subjects continue to
think of themselves according to the old paradigm. We believe that our
“personal” data belong to us, like our car or our shoes, and that
we’re only exercising our “individual freedom” by deciding to let
Google, Facebook, Apple, Amazon or the police have access to them,
without realizing that this has immediate effects on those who refuse
to, and who will be treated from then on as suspects, as potential
deviants. “To be sure,” predicts The New Digital Age, “there will be
people who resist adopting and using technology, people who want
nothing to do with virtual profiles, online data systems or smart
phones. Yet a government might suspect that people who opt out
completely have something to hide and thus are more likely to break
laws, and as a counterterrorism measure, that government will build
the kind of ‘hidden people’ registry we described earlier. If you
don’t have any registered social-networking profiles or mobile
subscriptions, and on-line references to you are unusually hard to
find, you might be considered a candidate for such a registry. You
might also be subjected to a strict set of new regulations that
includes rigorous airport screening or even travel restrictions.”

\subsection*{3. The Poverty of Cybernetics.}


So the security services are coming to consider a Facebook profile
more credible than the individual supposedly hiding behind it. This is
some indication of the porousness between what was still called the
virtual and the real. The accelerating datafication of the world does
make it less and less pertinent to think of the online world and the
real world, cyberspace and reality, as being separate. “Look at
Android, Gmail, Google Maps, Google Search. That’s what we do. We make
products that people can’t live without,” is how they put it in
Mountain View. In the past few years, however, the ubiquity of
connected devices in the everyday lives of human beings has triggered
some survival reflexes. Certain barkeepers decided to ban Google
Glasses from their establishments – which became truly hip as a
result, it should be said. Initiatives are blossoming that encourage
people to disconnect occasionally (one day per week, for a weekend, a
month) in order to take note of their dependence on technological
objects and re-experience an “authentic” contact with reality. The
attempt proves to be futile of course. The pleasant weekend at the
seashore with one’s family and without the smartphones is lived
primarily as an experience of disconnection; that is, as something
immediately thrown forward to the moment of reconnection, when it will
be shared on the Internet. Eventually, however, with Western man’s
abstract relation to the world becoming objectified in a whole complex
of apparatuses, a whole universe of virtual reproductions, the path
towards presence paradoxically reopens. By detaching ourselves from
everything, we’ll end up detaching ourselves even from our
detachment. The technological beatdown will ultimately restore our
capacity to be moved by the bare, pixelless existence of a honeysuckle
vine. Every sort of screen coming between us and reality will have
been required before we could reclaim the singular shimmer of the
sensible world, and our amazement at what is there. It will have taken
hundreds of “friends” who have nothing to do with us, “liking” us on
Facebook the better to ridicule us afterwards, for us to rediscover
the ancient taste for friendship. Having failed to create computers
capable of equaling human beings, they’ve set out to impoverish human
experience to the point where life can be confused with its digital
modeling. Can one picture the human desert that had to be created to
make existence on the social media seem desirable? Just as the
traveler had to be replaced by the tourist for it to be imagined that
the latter might pay to go all over the world via hologram while
remaining in their living room. But the slightest real experience will
shatter the wretchedness of this kind of illusionism. The poverty of
cybernetics is what will bring it down in the end. For a
hyper-individualized generation whose primary sociality had been that
of the social media, the Quebec student strike of 2012 was first of
all a stunning revelation of the insurrectionary power of simply being
together and starting to move. Evidently, this was a meet-up like no
other before, such that the insurgent friendships were able to rush
the police lines. The control traps were useless against that; in
fact, they had become another way for people to test themselves,
together. “The end of the Self will be the genesis of presence,”
envisioned Giorgio Cesarano in his Survival Manual. The virtue of the
hackers has been to base themselves on the materiality of the
supposedly virtual world. In the words of a member of Telecomix, a
group of hackers famous for helping the Syrians get around the state
control of Internet communications, if the hacker is ahead of his time
it’s because he “didn’t think of this tool [the Internet] as a
separate virtual world but as an extension of physical reality.” This
is all the more obvious now that the hacker movement is extending
itself outside the screens by opening hackerspaces where people can
analyze, tinker with, and piece together digital software and tech
objects. The expansion and networking of Do It Yourself has produced a
gamut of purposes: it’s a matter of fooling with things, with the
street, the city, the society, life itself. Some pathological
progressives have been quick to see the beginnings of a new economy in
it, even a new civilization, based this time on “sharing.” Never mind
that the present capitalist economy already values “creation,” beyond
the old industrial constraints. Managers are urged to facilitate free
initiative, to encourage innovative projects, creativity, genius, even
deviance – “the company of the future must protect the deviant, for
it’s the deviant who will innovate and who is capable of creating
rationality in the unknown,” they say. Today value is not sought in
the new features of a product, nor even in its desirability or its
meaning, but in the experience it offers to the consumer. So why not
offer that consumer the ultimate experience of going over to the other
side of the creation process? From this perspective, the hackerspaces
or “fablabs” become spaces where the “projects” of
“consumer-innovators” can be undertaken and “new marketplaces” can
emerge. In San Francisco, the TechShop firm is developing a new type
of fitness club where, for a yearly membership fee, “one goes every
week to make things, to create and develop one’s projects.” The fact
that the American army finances similar places under the Cyber Fast
Track program of DARPA (Defense Advanced Research Project Agency)
doesn’t discredit the hackerspaces as such. Any more than they’re
condemned to participate in yet another restructuring of the
capitalist production process when they’re captured in the “Maker”
movement with its spaces where people working together can build and
repair industrial objects or divert them from their original
uses. Village construction sets, like that of Open Source Ecology with
its fifty modular machines – tractor, milling machine, cement mixer,
etc. – and DIY dwelling modules could also have a different destiny
than serving to found a “small civilization with all the modern
comforts,” or creating “entire new economies” or a “financial system”
or a “new governance,” as its current guru fantasizes. Urban farming
which is being established on building roofs or vacant industrial
lots, like the 1300 community gardens of Detroit, could have other
ambitions than participating in economic recovery or bolstering the
“resilience of disaster zones.” Attacks like those conducted by
Anonymous/LulzSec against banking firms, security multinationals, or
telecommunications could very well go beyond cyberspace. As a
Ukrainian hacker says, “When you have to attend to your life, you stop
printing stuff in 3D rather quickly. You find a different plan.”

\subsection*{4. Techniques against Technology.}


The famous “question concerning technology,” still a blind spot for
revolutionary movements, comes in here. A wit whose name can be
forgotten described the French tragedy thus: “a generally technophobic
country dominated by a generally technophilic elite.”  While the
observation may not apply to the country, it does apply in any case to
the radical milieus. The majority of Marxists and post-Marxists
supplement their atavistic inclination to hegemony with a definite
attachment to technology-thatemancipates-man, whereas a large
percentage of anarchists and post-anarchists are down with being a
minority, even an oppressed minority, and adopt positions generally
hostile to “technology.” Each tendency even has its caricature:
corresponding to the Negriist devotees of the cyborg, the electronic
revolution by connected multitudes, there are the anti-industrials
who’ve turned the critique of progress and the “disaster of
technological civilization” into a profitable literary genre on the
whole, and a niche ideology where one can stay warm at least, having
envisaged no revolutionary possibility whatsoever. Technophilia and
technophobia form a diabolical pair joined together by a central
untruth: that such a thing as the technical exists. It would be
possible, apparently, to divide between what is technical and what is
not, in human existence. Well, no, in fact. One only has to look at
the state of incompletion in which the human offspring is born, and
the time it takes for it to move about in the world and to talk, to
realize that its relation to the world is not given in the least, but
rather the result of a whole elaboration. Since it’s not due to a
natural compatibility, man’s relation to the world is essentially
artificial, technical, to speak Greek. Each human world is a certain
configuration of techniques, of culinary, architectural, musical,
spiritual, informational, agricultural, erotic, martial, etc.,
techniques. And it’s for this reason that there’s no generic human
essence: because there are only particular techniques, and because
every technique configures a world, materializing in this way a
certain relationship with the latter, a certain form of life. So one
doesn’t “construct” a form of life; one only incorporates techniques,
through example, exercise, or apprenticeship. This is also why our
familiar world rarely appears to us as “technical”: because the set of
artifices that structure it are already part of us. It’s rather those
we’re not familiar with that seem to have a strange
artificiality. Hence the technical character of our world only stands
out in two circumstances: invention and “breakdown.”  It’s only when
we’re present at a discovery or when a familiar element is lacking, or
breaks, or stops functioning, that the illusion of living in a natural
world gives way in the face of contrary evidence.

Techniques can’t be reduced to a collection of equivalent instruments
any one of which Man, that generic being, could take up and use
without his essence being affected. Every tool configures and embodies
a particular relation with the world, and the worlds formed in this
way are not equivalent, any more than the humans who inhabit them
are. And by the same token these worlds are not hierarchizable
either. There is nothing that would establish some as more “advanced”
than others. They are merely distinct, each one having its own
potential and its own history. In order to hierarchize worlds a
criterion has to be introduced, an implicit criterion making it
possible to classify the different techniques. In the case of
progress, this criterion is simply the quantifiable productivity of
the techniques, considered apart from what each technique might
involve ethically, without regard to the sensible world it
engenders. This is why there’s no progress but capitalist progress,
and why capitalism is the uninterrupted destruction of
worlds. Moreover, the fact that techniques produce worlds and forms of
life doesn’t mean that man’s essence is production, as Marx
believed. So this is what technophiles and technophobes alike fail to
grasp: the ethical nature of every technique. It should be added that
the nightmare of this epoch is not in its being the “age of technics”
but in its being the age of technology. Technology is not the
consummation of technical development, but on the contrary the
expropriation of humans’ different constitutive techniques. Technology
is the systematizing of the most effective techniques, and
consequently the leveling of the worlds and the relations with the
world that everyone deploys. Techno-logy is a discourse about
techniques that is constantly being projected into material
reality. Just as the ideology of the festival is the death of the real
festival, and the ideology of the encounter is the actual
impossibility of coming together, technology is the neutralization of
all the particular techniques. In this sense capitalism is essentially
technological; it is the profitable organization of the most
productive techniques into a system. Its cardinal figure is not the
economist but the engineer. The engineer is the specialist in
techniques and thus the chief expropriator of them, one who doesn’t
let himself be affected by any of them, and spreads his own absence
from the world everywhere he can. He’s a sad and servile figure. The
solidarity between capitalism and socialism is confirmed there: in the
cult of the engineer. It was engineers who drew up most of the models
of the neoclassical economy like pieces of contemporary trading
software. Recall in this regard that Brezhnev’s claim to fame was to
have been an engineer in the metallurgical industry in Ukraine. The
figure of the hacker contrasts point by point with the figure of the
engineer, whatever the artistic, police-directed, or entrepreneurial
efforts to neutralize him may be. Whereas the engineer would capture
everything that functions, in such a way that everything functions
better in service to the system, the hacker asks himself “How does
that work?”  in order to find its flaws, but also to invent other
uses, to experiment. Experimenting then means exploring what such and
such a technique implies ethically. The hacker pulls techniques out of
the technological system in order to free them. If we are slaves of
technology, this is precisely because there is a whole ensemble of
artifacts of our everyday existence that we take to be specifically
“technical” and that we will always regard simply as black boxes of
which we are the innocent users. The use of computers to attack the
CIA attests rather clearly that cybernetics is no more the science of
computers than astronomy is the science of telescopes. Understanding
how the devices around us work brings an immediate increase in power,
giving us a purchase on what will then no longer appear as an
environment, but as a world arranged in a certain way and one that we
can shape. This is the hacker’s perspective on the world. These past
few years, the hacker milieu has gained some sophistication
politically, managing to identify friends and enemies more
clearly. Several substantial obstacles stand in the way of its
becoming-revolutionary, however. In 1986, “Doctor Crash” wrote:
“Whether you know it or not, if you are a hacker you are a
revolutionary. Don’t worry, you’re on the right side.” It’s not
certain that this sort of innocence is still possible. In the hacker
milieu there‘s an originary illusion according to which “freedom of
information,” “freedom of the Internet,” or “freedom of the
individual” can be set against those who are bent on controlling
them. This is a serious misunderstanding. Freedom and surveillance,
freedom and the panopticon belong to the same paradigm of
government. Historically, the endless expansion of control procedures
is the corollary of a form of power that is realized through the
freedom of individuals. Liberal government is not one that is
exercised directly on the bodies of its subjects or that expects a
filial obedience from them. It’s a background power, which prefers to
manage space and rule over interests rather than bodies. A power that
oversees, monitors, and acts minimally, intervening only where the
framework is threatened, against that which goes too far. Only free
subjects, taken en masse, are governed. Individual freedom is not
something that can be brandished against the government, for it is the
very mechanism on which government depends, the one it regulates as
closely as possible in order to obtain, from the amalgamation of all
these freedoms, the anticipated mass effect. Ordo ab chao. Government
is that order which one obeys “like one eats when hungry and covers
oneself when cold,” that servitude which I co-produce at the same time
that I pursue my happiness, that I exercise my “freedom of
expression.” “Market freedom requires an active and extremely vigilant
politics,” explained one of the founders of neoliberalism. For the
individual, monitored freedom is the only kind there is. This is what
libertarians, in their infantilism, will never understand, and it’s
this incomprehension that makes the libertarian idiocy attractive to
some hackers. A genuinely free being is not even said to be free. It
simply is, it exists, deploys its powers according to its being. We
say of an animal that it is en liberté, “roaming free,” only when it
lives in an environment that’s already completely controlled, fenced,
civilized: in the park with human rules, where one indulges in a
safari. “Friend” and “free” in English, and “Freund” and “frei” in
German come from the same Indo-European root, which conveys the idea
of a shared power that grows. Being free and having ties was one and
the same thing. I am free because I have ties, because I am linked to
a reality greater than me. In ancient Rome, the children of citizens
were liberi: through them, it was Rome that was growing. Which goes
to show how ridiculous and what a scam the individual freedom of “I do
what I feel like doing” is. If they truly want to fight the
government, the hackers have to give up this fetish. The cause of
individual freedom is what prevents them from forming strong groups
capable of laying down a real strategy, beyond a series of attacks;
it’s also what explains their inability to form ties beyond
themselves, their incapacity for becoming a historical force. A member
of Telecomix alerts his colleagues in these terms: “What is certain is
that the territory you’re living in is defended by persons you would
do well to meet. Because they’re changing the world and they won’t
wait for you.” Another obstacle for the hacker movement, as every new
meeting of the Chaos Computer Club demonstrates, is in managing to
draw a front line in its own ranks between those working for a better
government, or even the government, and those working for its
destitution. The time has come for taking sides. It’s this basic
question that eludes Julian Assange when he says: “We high-tech
workers are a class and it’s time we recognize ourselves as such.”
France has recently exploited the defect to the point of opening a
university for molding “ethical hackers”. Under DCRI supervision, it
will train people to fight against the real hackers, those who haven’t
abandoned the hacker ethic. These two problems merged in a case
affecting us. After so many attacks that so many of us applauded,
Anonymous/LulzSec hackers found themselves, like Jeremy Hammond,
nearly alone facing repression upon getting arrested. On Christmas
day, 2011, LulzSec defaced the site of Strafor, a “private
intelligence” multinational. By way of a homepage, there was now the
scrolling text of The Coming Insurrection in English, and \$700,000
was transferred from the accounts of Stratfor customers to a set of
charitable associations – a Christmas present. And we weren’t able to
do anything, either before or after their arrest. Of course, it’s
safer to operate alone or in a small group – which obviously won’t
protect you from infiltrators – when one goes after such targets, but
it’s disastrous for attacks that are so political, and so clearly
within the purview of global action by our party, to be reduced by the
police to some private crime, punishable by decades of prison or used
as a lever for pressuring this or that “Internet pirate” to turn into
a government snitch.

\vspace{10 mm}
Invisible Committee, October 2014

\end{document}
